%----------------------------------------------------------------------------------------
%	SECTION TITLE
%----------------------------------------------------------------------------------------

\cvsection{Research}

%----------------------------------------------------------------------------------------
%	SECTION CONTENT
%----------------------------------------------------------------------------------------

\begin{cventries}

%------------------------------------------------
\cventry
{EPFL, DCSL - Prof. Edouard Bugnion, Prof. James Larus, Prof. Mathias Payer}
{Current Research Area}
{Lausanne, Switzerland}
{Aug. 2019 - Present}
{ %TODO bridging gap between hardware features and programming languages
	\begin{cvitems}
  \item{Intersection between PL, systems, and Security}
	\item{Abusing existing programming abstractions to provide efficient support for
    security hardware extensions}
  \item{Information flow-control between code modules and in distributed systems}
  \item{Language and hardware-based isolation of mutually distrustful packages in applications}
	\end{cvitems}
}


\cventry
{EPFL, DSCL - Prof. Edouard Bugnion, Prof. James Larus}
  {Go \& SGX}
{Lausanne, Switzerland}
{Jun. 2018  - May 2019}
{
	\begin{cvitems}
	\item{Extended the goroutine abstraction and Go compiler to transparently
    support executing goroutines inside Intel SGX.}
	\item{Intel SGX, Confidentiality, Intergrity, Go, Compilers, Code partitioning, Hardware Extensions}
  \item{\texttt{Secured routines: Language-based construction of TEEs} (ATC19)}
	\end{cvitems}
}

\cventry
{EPFL, DSCL - Prof. Edouard Bugnion}
{Light-Weight Contexts in Dune}
{Lausanne, Switzerland}
{Sep. 2016 - Jul. 2017}
{
	\begin{cvitems}
  \item{Leveraged Dune to allow threads to switch between different views of
    an address space and take snapshots.\\Allows to protect secrets while executing untrusted library code, or to restore a prestine state after executing an RPC.}
  \item{5x speed improvement over a linux fork.}
	\item{Intel VTX, Dune, Virtualization, Kernel module, Virtual Memory Management}
	\end{cvitems}
}

\cventry
{Northeastern University - Prof. Jan Vitek} % Organization
{Efficient Runtime Deoptimization for R(Master Thesis)} % Job title
{Boston, U.S.A.} % Location
{Sep. 2015 - Mar. 2016} % Date(s)
{ % Description(s) of tasks/responsibilities
\begin{cvitems}
\item{Assumption-based optimizer for an R JIT compiler to remove performance bottlenecks inherent to the language, while preserving semantics at runtime.}
\item {On-stack replacement, assumption-based compiler optimizations, runtime deoptimization, R, LLVM, JIT compilers}
\end{cvitems}
}

\cventry
{ABB Corporate Research - Dr. Manuel Oriol}
{Aperiodic-Event Support in FASA}
{Baden, Switzerland}
{Feb. 2015 - Aug. 2015}
{
	\begin{cvitems}
	\item{Fixed-priority servers, data-driven events, real-time control applications, kernel design, dynamic linking/loading \& software updates, pi-calculus}
	\end{cvitems}
}

\cventry
{EPFL, LAMP - Prof. Martin Odersky \& Dr. Eugene Burmako}
{Scalameta: AST Persistence \& Obey: Code Health}
{Lausanne, Switzerland}
{Jan. 2014 - Feb. 2015}
{
	\begin{cvitems}
	\item{Obey: Scala-linter that accepts user-defined rules enforced at compile-time.}
	\item{AST Persistence: typed-AST based format for Scala code to resolve
    compiler version incompatibilities and facilitate maccros expansions in
    IDEs.}
	\end{cvitems}
}

\end{cventries}
