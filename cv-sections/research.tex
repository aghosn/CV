%----------------------------------------------------------------------------------------
%	SECTION TITLE
%----------------------------------------------------------------------------------------

\cvsection{Research \& Publications}

%----------------------------------------------------------------------------------------
%	SECTION CONTENT
%----------------------------------------------------------------------------------------

\begin{cvskills}

\cvskill
{Focus Areas}
{Systems, Kernels, Virtualization, Security, TEEs, Sandboxes\linebreak 
 Programming Abstractions, Compilers, Language Runtimes\linebreak
  Hardware-enforced isolation, Isolation of mutually distrustful software components\linebreak
}

\end{cvskills}

\begin{cventries}

%------------------------------------------------
\cventry
{Imperial College London: Marios Kogias, EPFL: Prof. Edouard Bugnion, Prof. Mathias Payer}
  {\href{https://aghosn.github.io/assets/pdf/tyche_aghosn.pdf}{Tyche: Creating Trust by Abolishing Hierarchies [HotOS 23]}}
{Cambridge, UK}
{Nov. 2021 - Present}
{ %TODO bridging gap between hardware features and programming languages
	\begin{cvitems}
  \item{Isolation monitor, hardware-independent support for compartmentalization \& confidential computing.}
  \item{Written in Rust, runs on x86 \& RISC-V}
  \item{Intel VT-x, Intel TXT, RISC-V PMP, Linux Kernel drivers, Virtualization}
	\end{cvitems}
}

\cventry
{EPFL: Charly Castes}
  {\href{https://dl.acm.org/doi/abs/10.1145/3623759.3624548}{Dynamic Linkers Are the Narrow Waist of Operating Systems [PLOS@SOSP 23]}}
{Cambridge, UK}
{Oct. 2023}
{ 
	\begin{cvitems}
  \item{Dynamic linker to port existing software to more secure execution environments.}
	\end{cvitems}
}

\cventry
{EPFL: 	Aleksander Boruch-Gruszecki, Mathias Payer, Clement Pit-Claudel}
  {\href{https://dl.acm.org/doi/10.1145/3689751}{Gradient: Gradual Compartmentalization via Object Capabilities Tracked in Types [OOSPLA24]}}
{Cambridge, UK}
{Oct. 2024}
{ 
	\begin{cvitems}
  \item{Gradual compartmentalization with object capabilities \& hardware-isolated compiled code.}
	\end{cvitems}
}


%\cventry
%{EPFL, DCSL - Prof. Edouard Bugnion, Prof. James Larus, Prof. Mathias Payer}
%{Ongoing Research}
%{Lausanne, Switzerland}
%{Aug. 2019 - Present}
%{ %TODO bridging gap between hardware features and programming languages
%	\begin{cvitems}
%  \item{Web-assembly as unit of intra-address space isolation}
%	\end{cvitems}
%}

\cventry
{EPFL - Prof. Edouard Bugnion, Prof. James Larus}
  {\href{https://infoscience.epfl.ch/record/289120}{PhD Thesis: Trust as a Programming Primitive}}
{Lausanne, Switzerland}
{Sep. 2016  - Sep. 2021}
{
	\begin{cvitems}
  \item{Programming Language extensions for compartmentalization and confidential computing.}
  \item{Programming languages, isolation, security, confidentiality, integrity, virtualization, hardware security extensions}
 % \item{\href{https://asplos-conference.org/papers/}{\textbf{ASPLOS link}}}
	\end{cvitems}
}


\cventry
{EPFL - Prof. Edouard Bugnion, Prof. Mathias Payer}
  {\href{https://aghosn.github.io/assets/pdf/enclosure_aghosn.pdf}{Enclosures: Language-based restriction of untrusted libraries [ASPLOS21]}}
{Lausanne, Switzerland}
{Sep. 2019  - Oct. 2020}
{
	\begin{cvitems}
  \item{New fine-grain programming abstraction to restrict public libraries access to program resources}
  \item{Frontend extensions to Go and Python PLs, backend hardware isolation enforcement (Intel VT-x \& Intel MPK)}
  \item{Intra-address-space isolation, Sandboxing, Compiler, Linker, Runtime}
 % \item{\href{https://asplos-conference.org/papers/}{\textbf{ASPLOS link}}}
	\end{cvitems}
}

\cventry
{EPFL - Prof. Edouard Bugnion, Prof. James Larus}
  {\href{https://www.usenix.org/system/files/atc19-ghosn_0.pdf}{Secured Routines: Language-based construction of TEEs [ATC19]}}
{Lausanne, Switzerland}
{Jun. 2018  - May 2019}
{
	\begin{cvitems}
  \item{Extended Go programming language to  support executing goroutines inside Intel SGX.}
	\item{Intel SGX, Confidentiality, Intergrity, Go, Compilers, Code partitioning, Hardware Extensions}
  %\item{\href{https://www.usenix.org/conference/atc19/presentation/ghosn}{\textbf{USENIX link}}}
	\end{cvitems}
}

\cventry
{EPFL - Prof. Edouard Bugnion}
{Light-Weight Contexts in Dune}
{Lausanne, Switzerland}
{Sep. 2016 - Jul. 2017}
{
	\begin{cvitems}
  \item{Process virtualization with Dune}
  \item{Intra-address space isolation, protecting secrets, memory snapshots, 5x faster than fork}
%  \item{5x speed improvement over a Linux fork}
	\item{Intel VTX, Dune, Virtualization, Kernel module, Virtual Memory Management}
	\end{cvitems}
}

\cventry
{Northeastern University - Prof. Jan Vitek} % Organization
{Efficient Runtime Deoptimization for R(Master Thesis)} % Job title
{Boston, U.S.A.} % Location
{Sep. 2015 - Mar. 2016} % Date(s)
{ % Description(s) of tasks/responsibilities
\begin{cvitems}
\item{Speculative optimizer for an R JIT compiler}
\item{Removes performance bottlenecks due to the language semantics}
%\item{Ensures correct run-time behavior.}
\item {On-stack replacement, speculative optimizations, runtime de-optimization, R, LLVM, JIT compiler}
\end{cvitems}
}

%\cventry
%{ABB Corporate Research - Dr. Manuel Oriol}
%{Aperiodic-Event Support in FASA}
%{Baden, Switzerland}
%{Feb. 2015 - Aug. 2015}
%{
%	\begin{cvitems}
%  \item{Fixed-priority servers, data-driven events, real-time control applications}
%  \item{kernel design, dynamic linking/loading \& software updates, pi-calculus}
%	\end{cvitems}
%}

\cventry
{EPFL, LAMP - Prof. Martin Odersky \& Dr. Eugene Burmako}
{Scalameta: AST Persistence \& Obey: Code Health}
{Lausanne, Switzerland}
{Jan. 2014 - Feb. 2015}
{
	\begin{cvitems}
	\item{Obey: Scala-linter for user-defined rules enforced at compile-time}
  \item{AST Persistence: typed-AST format for Scala for compiler version compatibility \& macro expansion}
 % \item{Resolves compiler version incompatibilities and provides IDE macros expansion support}
	\end{cvitems}
}

\cventry
{Undergraduate} % Affiliation/role
{Operating Systems \& Design 15-410} % Organization/group
{CMU} % Location
{Jan. 2013 - Jul. 2013} % Date(s)
{ % Description(s) of experience/contributions/knowledge
\begin{cvitems}
%\item {Implementation of a x86 Unix like Kernel in C and ASM}
%\item{Design and implementation of thread library, scheduler, virtual memory, various drivers, system calls}
\item{Design \& implementation of x86 Unix kernel -- thread library, scheduler, virtual memory, drivers, syscalls}
\end{cvitems}
}

\end{cventries}
