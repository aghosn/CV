% !TEX TS-program = xelatex
% !TEX encoding = UTF-8

%%%%%%%%%%%%%%%%%%%%%%%%%%%%%%%%%%%%%%%%%%%%%%%%%%%%%%%%%%%%%%%%%
%% SIMPLE-RESUME-CV
%% <https://github.com/zachscrivena/simple-resume-cv>
%% This is free and unencumbered software released into the
%% public domain; see <http://unlicense.org> for details.
%%%%%%%%%%%%%%%%%%%%%%%%%%%%%%%%%%%%%%%%%%%%%%%%%%%%%%%%%%%%%%%%%

%%%%%%%%%%%%%%%%%%%%%%%%%%%%%%%%%%%%%%%%%%%%%%%%%%%%%%%%%%%%%%%%%
%% INSTRUCTIONS FOR COMPILING THIS DOCUMENT ("CV.tex")
%% TEX ---(XeLaTeX)---> PDF:
%%
%% Method 1: Use latexmk for fully automated document generation:
%%   latexmk -xelatex "CV.tex"
%%   (add the -pvc switch to automatically recompile on changes)
%%
%% Method 2: Use XeLaTeX directly:
%%   xelatex "CV.tex"
%%   (run multiple times to resolve cross-references if needed)
%%%%%%%%%%%%%%%%%%%%%%%%%%%%%%%%%%%%%%%%%%%%%%%%%%%%%%%%%%%%%%%%%

\documentclass[a4paper,10pt,oneside]{article}
%\documentclass[letterpaper,10pt,oneside]{article}

% Do not stop on errors during compilation.
\nonstopmode

%%%%%%%%%%%%%%%%%%%%%%%%%%%%%%%%%%%%%%%%%%%%%%%%%%%%%%%%%%%%%%%%%
%% PREAMBLE.
%%%%%%%%%%%%%%%%%%%%%%%%%%%%%%%%%%%%%%%%%%%%%%%%%%%%%%%%%%%%%%%%%

%%%%%%%%%%%%%%%%%%%%%%%%%%%%%%%%%%%%%%%%%%%%%%%%%%%%%%%%%%%%%%%%%
%% SIMPLE-RESUME-CV
%% <https://github.com/zachscrivena/simple-resume-cv>
%% This is free and unencumbered software released into the
%% public domain; see <http://unlicense.org> for details.
%%%%%%%%%%%%%%%%%%%%%%%%%%%%%%%%%%%%%%%%%%%%%%%%%%%%%%%%%%%%%%%%%

% Long table for page layout.
\usepackage{longtable}

% Geometry package for page margins.
\usepackage[
left=0.70in,
right=0.70in,
top=0.70in,
bottom=0.55in,
nohead,
includefoot]{geometry}

% Hyphenation: Disabled.
\usepackage[none]{hyphenat}

% XeLaTeX packages.
\usepackage{fontspec}
\defaultfontfeatures{Ligatures=TeX}
\usepackage{xunicode}
\usepackage{xltxtra}

% Font: Use "Tinos" as the main typeface (\textnormal{}, \normalfont).
% The "Tinos" fonts are released under the Apache License Version 2.0,
% and can be downloaded for free at <http://www.fontsquirrel.com/fonts/tinos>.
% Symbol table: <http://www.fileformat.info/info/unicode/font/tinos/grid.htm>
\setmainfont
[Path=./Fonts/Tinos/,
ItalicFont=Tinos-Italic,
BoldFont=Tinos-Bold,
BoldItalicFont=Tinos-BoldItalic]
{Tinos-Regular.ttf}

% Sans-serif font: Switched to "Tinos".
\renewcommand{\sffamily}{\rmfamily}

% Typewriter (monospace) font: Switched to "Tinos".
\renewcommand{\ttfamily}{\rmfamily}

% Small caps font: Switched to "Tinos".
\renewcommand{\scshape}{\rmfamily}

% Secondary font: "GNU FreeFont".
% The "GNU FreeFont" fonts are released under the
% GNU General Public License Version 3, and can be downloaded
% for free at <https://savannah.gnu.org/projects/freefont/>.
\newcommand{\UseSecondaryFont}{\fontspec
[Path=./Fonts/GNUFreeFont/,
ItalicFont=FreeSerifItalic,
BoldFont=FreeSerifBold,
BoldItalicFont=FreeSerifBoldItalic]
{FreeSerif.otf}}

% Symbols (unicode).
\newcommand{\BulletSymbol}{{\char"2022}}
\newcommand{\TildeSymbol}{{\char"007E}}

% PDF settings and properties.
\usepackage{hyperref}

% Headers and footers: Blank header, page number in footer.
\makeatletter
\def\ps@plain{%
\def\@oddhead{}%
\def\@evenhead{}%
\def\@oddfoot{\footnotesize\hfill{Page}~{\thepage}~of~\pageref{LastPage}\hfill}%
\def\@evenfoot{\footnotesize\hfill{Page}~{\thepage}~of~\pageref{LastPage}\hfill}}
\makeatother

\pagestyle{plain}

% Paragraph style: No indentation.
\setlength{\parindent}{0in}

% Footnotes: Use symbols instead of numbers for labels.
\renewcommand{\thefootnote}{\fnsymbol{footnote}}

% Current date and time.
\usepackage[yyyymmdd,24hr]{datetime}
\renewcommand{\dateseparator}{-}
% {\today}~{\currenttime}

% Abbreviations for months.
\newcommand{\LongMonth}[1]{%
\ifcase#1\relax
\or January%
\or February%
\or March%
\or April%
\or May%
\or June%
\or July%
\or August%
\or September%
\or October%
\or November%
\or December%
\fi}
\newcommand{\ShortMonth}[1]{%
\ifcase#1\relax
\or Jan%
\or Feb%
\or Mar%
\or Apr%
\or May%
\or Jun%
\or Jul%
\or Aug%
\or Sep%
\or Oct%
\or Nov%
\or Dec%
\fi}

% Select datestamp format.
\def\DatestampFormatSelection{2}

% Datestamp format: {yyyy}{MM}{dd} ---> yyyy-MM-dd (e.g., 2010-12-31).
\ifnum\DatestampFormatSelection=1
\newcommand{\DatestampYMD}[3]{\mbox{#1-#2-#3}}
\newcommand{\DatestampYM}[2]{\mbox{#1-#2}}
\newcommand{\DatestampY}[1]{#1}
\fi

% Datestamp format: {yyyy}{MM}{dd} ---> MMM yyyy (e.g., Dec 2010).
\ifnum\DatestampFormatSelection=2
\newcommand{\DatestampYMD}[3]{\mbox{\ShortMonth{#2} #1}}
\newcommand{\DatestampYM}[2]{\mbox{\ShortMonth{#2} #1}}
\newcommand{\DatestampY}[1]{#1}
\fi

% Datestamp format: {yyyy}{MM}{dd} ---> MMMM yyyy (e.g., December 2010).
\ifnum\DatestampFormatSelection=3
\newcommand{\DatestampYMD}[3]{\mbox{\LongMonth{#2} #1}}
\newcommand{\DatestampYM}[2]{\mbox{\LongMonth{#2} #1}}
\newcommand{\DatestampY}[1]{#1}
\fi

% Datestamp format: {yyyy}{MM}{dd} ---> yyyy (e.g., 2010).
\ifnum\DatestampFormatSelection=4
\newcommand{\DatestampYMD}[3]{#1}
\newcommand{\DatestampYM}[2]{#1}
\newcommand{\DatestampY}[1]{#1}
\fi

% Macro: title (name).
\renewcommand{\title}[1]{%
\pdfbookmark[1]{#1}{#1}%
\par\begin{center}%
\par\begin{Huge}%
\textbf{#1}%
\par\end{Huge}%
\par\end{center}%
\par\vspace{-1.75em}\par}

% Macro: subtitle (personal information below name).
\newenvironment{subtitle}
{\par\begin{center}%
\par\begin{footnotesize}}
{\par\end{footnotesize}%
\par\end{center}\par}

% Macro: body (rest of the document).
\newenvironment{body}
{\par\vspace{-1em}\par
\begin{longtable}{p{0.15\textwidth}p{0.80\textwidth}}}
{\par\end{longtable}\par}

% Macro: section (new section for Education, Research Experience, etc.).
\renewcommand{\section}[3]{\\[-1em]\pdfbookmark[2]{#2}{#3}\\%
{\fontsize{9pt}{11pt}\selectfont\bfseries\raggedright%
\MakeUppercase{#1}}&}

% Macro: subsection.
\renewcommand{\subsection}[3]{\par\vskip-\baselineskip%
\pdfbookmark[3]{#2}{#3}\par%
{\fontsize{8pt}{9.6pt}\selectfont\bfseries\raggedright%
\MakeUppercase{#1}}%
\vspace{0.1em}}

% Macro: EntryGap (vertical gap between entries in the same section).
\newcommand{\EntryGap}{\\[-0.5em]~&}

% Macro: SmallEntryGap (small vertical gap within a long entry).
\newcommand{\SmallEntryGap}{\par\vspace{0.25em}\par}

% Macro: BibSpace (small horizontal space in bibliographic entries).
\newcommand{\BibSpace}{\hspace{0.5em}\ignorespaces}

% Macro: detail (text in smaller font under an entry).
\newenvironment{detail}
{\par\begingroup\fontsize{8.6}{10.3}\selectfont}
{\par\endgroup\par}

% Macro: list of skills
\newenvironment{skilling}
{\par\begingroup\fontsize{8.6}{10.3}\selectfont}
{\par\endgroup\par}

% Macro: BulletItem.
\newcommand{\BulletItem}{\par%
\noindent\hangafter=1\hangindent=1.55em\ignorespaces%
\hspace{0.8em}\BulletSymbol\hspace{0.4em}\ignorespaces}

% Macro: NumberedItem.
\newcommand{\NumberedItem}[1]{\par%
\noindent\hangafter=1\hangindent=1.75em\ignorespaces%
{#1)}\hspace{0.4em}\ignorespaces}

% Macro: hide.
\newcommand{\hide}[1]{}


% CV Info (to be customized).
\newcommand{\CVTitle}{Adrien Ghosn's CV}

% PDF settings and properties.
\hypersetup{
pdftitle={\CVTitle},
pdfauthor={Adrien Ghosn},
pdfsubject={CV},
pdfcreator={XeLaTeX},
pdfproducer={},
pdfkeywords={CV, Adrien Ghosn, Computer Science, Programming, Java, Scala},
pdfpagemode={},
bookmarks=true,
unicode=true,
bookmarksopen=true,
pdfstartview=FitH,
pdfpagelayout=OneColumn,
pdfpagemode=UseOutlines,
hidelinks,
breaklinks}

%%%%%%%%%%%%%%%%%%%%%%%%%%%%%%%%%%%%%%%%%%%%%%%%%%%%%%%%%%%%%%%%%
%% ACTUAL DOCUMENT.
%%%%%%%%%%%%%%%%%%%%%%%%%%%%%%%%%%%%%%%%%%%%%%%%%%%%%%%%%%%%%%%%%

\begin{document}
\clearpage
\thispagestyle{empty}
%%%%%%%%%%%%%%%
% TITLE BLOCK %
%%%%%%%%%%%%%%%

\title{Adrien Ghosn}

\begin{subtitle}
\href{https://www.google.com/maps/place/6+Rue+de+la+Saint-Martin,+74160+Saint-Julien-en-Genevois,+France/@46.1462523,6.0852366,17z/data=!3m1!4b1!4m2!3m1!1s0x478c7b8a6860d24b:0xb4b59d1320abf78e}
{6a rue de la Saint Martin, Apt C01, Saint-Julien en Genevois, Haute-Savoie 74160, FRANCE}
\par
\href{mailto:ghosn.adrien@gmail.com}
{ghosn.adrien@gmail.com}
\,\BulletSymbol\,
+33\,6\,13\,17\,61\,55
\,\BulletSymbol\,
\href{https://github.com/aghosn}
{github.com/aghosn}
\,\BulletSymbol\,
\href{http://ch.linkedin.com/in/aghosn}
{ch.linkedin.com/in/aghosn}
\end{subtitle}


\begin{body}

%%%%%%%%%%%%%%%
%% EDUCATION %%
%%%%%%%%%%%%%%%

\section
{Education}
{Education}
{PDF:Education}

\href{https://www.epfl.ch/}
{\textbf{Ecole Polytechnique Federale de Lausanne (EPFL)}},
Lausanne, Switzerland
\SmallEntryGap
PhD in Computer Science, \href{http://dcsl.epfl.ch/}{DCSL}
\hfill
\DatestampYMD{2016}{09}{01} --
Present

\SmallEntryGap
Master Degree in
\href{http://ic.epfl.ch/computer-science/master}
{Computer Science Engineering}
\hfill
\DatestampYMD{2013}{09}{15} --
\DatestampYMD{2016}{04}{15}
\begin{detail}
\BulletItem
Specialisation:
\href{http://ic.epfl.ch/specializations}
{Foundations of Software}, advised by Prof. M Odersky
\BulletItem
Average:
5.75 / 6.00
\end{detail}

\SmallEntryGap
Bachelor Degree in
\href{http://ic.epfl.ch/computer-science/bachelor}
{Computer Science Engineering}
\hfill
\DatestampYMD{2010}{09}{15} --
\DatestampYMD{2013}{07}{15}

\EntryGap
\href{http://www.cmu.edu/}
{\textbf{Carnegie Mellon University (CMU)}},
Pittsburgh, Pennsylvania, USA
\par
Exchange Year in Bachelor Degree in Computer Science
\hfill
\DatestampYMD{2012}{08}{15} --
\DatestampYMD{2013}{07}{15}
\begin{detail}
\BulletItem
Dean's list, School of Computer Science for QPA > 3.75 / 4.00
\end{detail}

\hline

%%%%%%%%%%%%
%% SKILLS %%
%%%%%%%%%%%%

\section
{Skills}
{Skills}
{PDF:Skills}

\textbf{Programming Languages}
\par
\begin{skilling}
Java, 
Scala,
C, 
C++, 
Shell Scripting,
Python, 
asm,
Haskell,
Perl,
JavaScript (\& HTML/CSS),
SQL
\end{skilling}

\EntryGap
\textbf{Tools \& Other skills}
\par
\begin{skilling}
Git, SVN, Eclipse, VIM, SBT, PlayFramework Java/Scala, TOMCAT,
Relational Databases, OpenStreet Map, Google/Twitter's APIs, Hadoop, Spark, Map/Reduce,
TCP/IP Networking, IT Security, Cryptography, Model Based System Design, Theoretical Computer Science,
Concurrent \& Distributed algorithms.
\end{skilling}

\hline
%%%%%%%%%%%%%%%%%%%%%%%%%
%% RESEARCH EXPERIENCE %%
%%%%%%%%%%%%%%%%%%%%%%%%%

\section
{Research Experience}
{Research Experience}
{PDF:ResearchExperience}

{\textbf{EPFL, DCSL}},
PhD Student
\hfill
\DatestampYMD{2016} --
\DatestampYMD{Present}
\begin{detail}
\BulletItem
Project:
Light-weight contexts in Dune
\BulletItem
Supervisors:
Prof Edouard Bugnion
\BulletItem
Research areas:
Light-weight contexts allowing processes to create and switch among different address spaces in a virtualized environment (intel VTX).
\end{detail}

\href{http://http://www.ccs.neu.edu/research/prl/}
{\textbf{Northeastern University, Boston}},
Master Thesis Student
\hfill
\DatestampYMD{2015}{09}{15} --
\DatestampYMD{2016}{03}{15}
\begin{detail}
\BulletItem
Project:
Efficient runtime deoptimization for R 
\BulletItem
Supervisors:
Prof Jan Vitek and
Prof Viktor Kuncak
\BulletItem
Research areas:
On-stack replacement, assumption-based compiler optimizations, R, LLVM, JIT compilers
\end{detail}

\href{http://www.abb.com/}
{\textbf{ABB Corporate Research}},
Graduate Research Intern
\hfill
\DatestampYMD{2015}{02}{15} --
\DatestampYMD{2015}{08}{15}
\begin{detail}
\BulletItem
Project:
Aperiodic Support in FASA
\BulletItem
Supervisors:
Dr. Manuel Oriol and
Dr. Aurelien Monot
\BulletItem
Research areas:
Fixed-priority servers, data-driven events, real-time control applications, Kernel, dynamic linking/loading, $\pi$-calculus 
\end{detail}

\EntryGap
\href{http://lamp.epfl.ch/}
{\textbf{EPFL Programming Methods Laboratory (LAMP)}},
Graduate
\hfill
\DatestampYMD{2014}{09}{15} --
\DatestampYMD{2015}{02}{15}
\begin{detail}
\BulletItem
Project:
\href{http://infoscience.epfl.ch/record/204804?ln=en}
{Obey, Code health for Scalameta}
\BulletItem
Supervisors:
Prof. M. Odersky and 
Eugene Burmako
\BulletItem
Description:
Auto-correction and formatting at compile-time of Scala source code, according to user-defined rules.
The project enables to automatically correct the source code to comply with project-defined formats or adapt to new library interfaces.
\end{detail}

\EntryGap
\href{http://lamp.epfl.ch/}
{\textbf{EPFL Programming Methods Laboratory (LAMP)}},
Graduate
\hfill
\DatestampYMD{2014}{01}{15} --
\DatestampYMD{2014}{07}{15}
\begin{detail}
\BulletItem
Project:
\href{http://infoscience.epfl.ch/record/200050?ln=en}
{AST Persistence for Scalameta}
\BulletItem
Supervisors:
Prof. M. Odersky and 
Eugene Burmako
\BulletItem
Description:
Efficiently storing a compressed version of Scala Abstract Syntax Trees (AST's).
This new format resolves compiler version incompatibilities for Scala libraries, while containing more information than the .jar compiled byte-code.
\end{detail}

%%%%%%%%%%%%%%%%%%%%%%%%%%%%%%%%%%%%%%%%%%%%
%% PROJECTS & RELEVANT COURSES %%
%%%%%%%%%%%%%%%%%%%%%%%%%%%%%%%%%%%%%%%%%%%%

\section
{PROJECTS}
{Projects}
{PDF:Projects}

\href{https://www.cs.cmu.edu/~410/}
{\textbf{Operating Systems Implementation \& Design}}, 15-410 CMU
\hfill
\DatestampYMD{2013}{01}{15} --
\DatestampYMD{2013}{07}{15}
\begin{detail}
\BulletItem
Description:
Implementation of a x86 Unix like Kernel in C and ASM. The project required to design and implement the thread library, the virtual memory, the drivers for the display, keyboard and clock, the system calls and an efficient scheduler.
\end{detail}

\EntryGap
\href{http://crossstream.ch/}
{\textbf{Tweet Aggregator}}, EPFL
\hfill
\DatestampYMD{2014}{01}{15} --
\DatestampYMD{2014}{07}{15}
\begin{detail}
\BulletItem
Description:
Big Data web application that gathers and displays real-time tweets according to user-defined keywords.
The application gives a fine-grained filtering of tweets according to zoom-level and selected geographical areas. The project evolved into crossstream.ch.
\end{detail}

\EntryGap
\textbf{Compiler \& Advanced Compiler}, EPFL
\hfill
\DatestampYMD{2013}{09}{15} --
\DatestampYMD{2014}{07}{15}
\begin{detail}
\BulletItem
Description:
Compilers for Java-like and Lisp-like languages. Implementation of optimizations such as DCE-CSE-Constant Folding-Closures-Hoisting and a garbage collector.
\end{detail}

\hline
%%%%%%%%%%%%%%%
%% LANGUAGES %%
%%%%%%%%%%%%%%%

\section
{Languages}
{Languages}
{PDF:Languages}

French (Native language), English (Fluent), Italian (Notions)
\end{body}

%%%%%%%%%%%
% CV NOTE %
%%%%%%%%%%%

\end{document}
